This chapter classifies the system in terms of successes, failures, evaluation, and future development for an inventory check. While making the system, I have gained a lot of useful knowledge, skills, and experience when making software by my self. When deploying the system to my uncle and other shopkeepers, I have received so many positive feedback via message.  

\section{System Successes}
Overall the project has been successful where design, technical requirements, and system development meet all the requirements that were set out. The main success was using a system to store the data in the database are met.\newline 
\newline The implementation of the front-end allows the user to interact throughout the site with a number of different features. This includes the right choice of color, easy to navigate around the site, the best bar chart to show relevant data, and easy contact options to the developer. The system loads up very quickly when the user has a stable internet connection of less than 1Mbps speed. This was one of the successes with the front-end development.\newline
\newline The implementation of the back-end server using PHP Script and MySQL to interact with the database to retrieve and store the data. The back-end has all the ability to store all kinds of data such as storing products in the database. All the features for back-end service are working perfectly such as view all products, add products, edit products, delete the product, and show this data in chart-wise. The contact form has Telegram API ability to send the message in developer telegram messenger app.\newline
\newline In the user survey, inventory check was satisfied with the result received through the survey. This was very helpful which gave me an understanding of what their requirement is. The majority of people love to use my system. By gathering all this data has helped me to make a successful system.\newline
\newline During the user testing, I have received a lot of positive feedback for my system. It has found that the application is easy to navigate and use throughout the site. The tester has no difficulties with all the features that are currently available within the system. The user can access the map showing the location on the contact page where they can see the developer location with a contact form as well.

\section{System Failures}
There are some failures in the system which cannot be resolved due to time constraint. There are many features that are not has been implemented with this version of the software but will be implemented in future developments. The following list is the system failures that were found during the user testing and normal testing.

\subsection{Register}
User has found that they cannot able to see the create new account page. There was an error with the code due to the bootstrap 4 library model. They have also found that the cannot able to register the new user as this feature has not been implemented with the database.

\subsection{Login}
User has found that they cannot able to login into the system. The system currently does not have this feature where the user can log in to the system. They can access all the pages without login credentials.

\subsection{Mobile Friendly (Responsive)}
The user has found that some of the website pages are not mobile-friendly especially the homepage. This feature could be solved by using media queries for different viewport so the user can view the website in any device without any difficulties.

\subsection{No alert for Internet connection}
User has found that they cannot see any alert message for no internet connection. This feature has not been added in the system but this can be in future development for reminding the user to turn on the internet.

\subsection{Alert for Delete product}
The user has found that there is no confirmation box when deleting the product from the database. This feature has been implemented in the system but this could be done using JavaScript alert command. 

\section{Future Developments}
Inventory check is ready to deploy in the market as a finished product but there are some features that will come in future development. The following list will be built in future developments.

\subsection{Login \& Register}
Currently, The login and register form has been build where user can add their credential information to login into the system but the functionality for back-end system will be built in future development where all the pages will be locked until user login into the system. When making the login and register feature, I will need to use a password hashing method to make sure that login credentials are secure on the server-side and protected from the hackers. In the database, the password will be encrypted with the hashing method which makes it complex for a malicious person to view the password. 

\subsection{More charts}
There are only two charts are available in the system for now where user can analyse the data. In the future, my plan is to build a customisation feature where the user chooses the list of the data which will then convert into the charts.

\subsection{Filter Features}
I have tried to add this feature in view of all product pages where users can filter the product in ascending order, descending order, time, stock, quantity, remaining stock, etc. But due to time constrain I have decided to add this feature in future development. Currently, The user can use search functionality until the filter features are added in the next application update.

\subsection{POS system}
This will be one of the most useful features to add in the future update where the database will update automatically with POS (Point of Sale). This system will update the stock availability, sold item, etc.

\subsection{Barcoding}
This feature will help the user to not making a mistake when adding a product in the database. They can add an item by scanning the barcode to give reliable and fast solutions to record the item. Which this feature they can easily find the stock availability and price by scanning the barcode of the item.

\subsection{Alerts}
By adding a feature in the system will give alert to the user via on-screen notification when the item is low in stock, daily sales, and any error in the system. This will help the user to find the issues more quickly and receive alerts.

\subsection{Responsive site}
Currently, there are some pages that 
is not mobile-friendly which was found in user testing by my friend. This will be fixed using the media query in future development so the user can view the website on their mobile without any problem.

\subsection{Alert for delete product}
While doing the unit testing, the website does not give a confirmation dialog box when deleting the product. This feature will be added in a later version where the system asks for confirmation users such as "Do you want to delete this product" and users need to select by two given options "YES" or "NO".

\subsection{PDF file}
This feature will allow the user to download all products in PDF file where they can find useful items such as low stock items, reaming stock, etc.

\subsection{Protect user data}
Inventory check application holds user data in the database server and not fully protected. There will be a risk for data theft so in future versions all the important data will be encrypted to save from hackers.

\subsection{Required hardware}
Many features will come in future development will require specific hardware to interact with the database such as POS till for automatic update the sold item, scanning and barcoding mechanism for adding item or view item information. 

\section{Evaluation}
This project was aimed to tackle inventory management in local retail shopkeepers. Inventory check was selected to enhance the issues that they are currently facing and providing them with the best system for free. Stakeholder gathering was valuable which laid out the bunch of problems that they are currently facing with their current system. This gave me an idea of what their requirement is with the new system. Having strong planning across the project was the main success of the project to finished on time. When I fall sick, I had to manage my time more efficiently and finish the outstanding task to keep my self on target. During the design stage of the project, I need to ensure that stakeholder is happy with the design.\newline
\newline It was a complex system to build that uses three various technologies such as Front-end, back-end, and MySQL database. This was the biggest achievement I have made when making complex software with limited time on my own. It was a fun project to make which will helps hundred of a shopkeeper who use the spreadsheet to manage their inventory and I am happy that I have solved their problem by making this application. I have faced a lot of problems when building this application such as learning new technologies, fixing the problem within the time specified in the Gantt chart, fixing the database server connection with PHP script, etc. Using Gantt chart was allowed me to manage my time and finished the project on time.\newline
\newline I have never thought at the start of the project that I would build a complicated system, But now I am happy that I have made this system.  